\documentclass[12pt,a4paper]{article}

\usepackage[right=2cm,left=2cm,top=2cm,bottom=2cm]{geometry}
\usepackage{luatextra}
\usepackage[french]{babel}
\usepackage{fontspec}
\setmainfont{Linux Libertine O}
\usepackage{xcolor}
\usepackage{hyperref}
	\hypersetup{
		colorlinks=true,
		linkcolor=black,
	}
\usepackage{fancyhdr}
\pagestyle{empty}

\setlength\parindent{0pt}

\begin{document}

\begin{Form}

\begin{center}
	\textbf{\uppercase{Attestation de déplacement dérogatoire}} \\
	En application de l'article 3 du décret du 23 mars 2020 prescrivant les mesures générales
	nécessaires pour faire face à l'épidémie de Covid19 dans le cadre de l'état d'urgence sanitaire 
\end{center}

Je soussigné(e), \\

\TextField[width=5cm,borderwidth=0pt]{Mme/M. :} \\
\TextField[width=5cm,borderwidth=0pt]{Né(e) le :} \\
\TextField[width=5cm,borderwidth=0pt]{À :} \\
\TextField[width=14cm,borderwidth=0pt]{Demeurant :} \\

certifie que mon déplacement est lié au motif suivant (cocher la case) autorisé par l'article 3
du décret du 23 mars 2020 prescrivant les mesures générales nécessaires pour faire face à
l'épidémie de Covid19 dans le cadre de l'état d'urgence sanitaire\footnote{Les personnes
souhaitant bénéficier de l'une de ces exceptions doivent se munir s'il y a lieu, lors de leurs
déplacements hors de leur domicile, d'un document leur permettant de justifier que le déplacement
considéré entre dans le champ de l'une de ces exceptions.} :\\

\begin{minipage}{0.1\textwidth}
	\CheckBox[bordercolor=black]{}
\end{minipage}
\begin{minipage}{0.89\textwidth}
	Déplacements entre le domicile et le lieu d'exercice de l'activité professionnelle,
	lorsqu'ils sont indispensables à l'exercice d'activités ne pouvant être organisées sous
	forme de télétravail ou déplacements professionnels ne pouvant être différés\footnotemark.
\end{minipage}
\footnotetext{À utiliser par les travailleurs non-salariés, lorsqu'ils ne peuvent disposer d'un 
	justificatif de déplacement établi par leur employeur.}

\vspace{5mm}

\begin{minipage}{0.1\textwidth}
	\CheckBox[bordercolor=black]{}
\end{minipage}
\begin{minipage}{0.89\textwidth}
	Déplacements pour effectuer des achats de fournitures nécessaires à l’activité
	professionnelle et des achats de première nécessité\footnotemark{} dans des établissements dont les
	activités demeurent autorisées (liste sur gouvernement.fr).
\end{minipage}
\footnotetext{Y compris les acquisitions à titre gratuit (distribution de denrées alimentaires\dots) et
	les déplacements liés à la perception de prestations sociales et au retrait d'espèces.}
	
\vspace{5mm}

\begin{minipage}{0.1\textwidth}
	\CheckBox[bordercolor=black]{}
\end{minipage}
\begin{minipage}{0.89\textwidth}
	Consultations et soins ne pouvant être assurés à distance et ne pouvant être différés ;
	consultations et soins des patients atteints d'une affection de longue durée.
\end{minipage}

\vspace{5mm}

\begin{minipage}{0.1\textwidth}
	\CheckBox[bordercolor=black]{}
\end{minipage}
\begin{minipage}{0.89\textwidth}
	Déplacements pour motif familial impérieux, pour l'assistance aux personnes
	vulnérables ou la garde d’enfants.
\end{minipage}

\vspace{5mm}

\begin{minipage}{0.1\textwidth}
	\CheckBox[bordercolor=black]{}
\end{minipage}
\begin{minipage}{0.89\textwidth}
	Déplacements brefs, dans la limite d'une heure quotidienne et dans un rayon maximal
	d'un kilomètre autour du domicile, liés soit à l'activité physique individuelle des
	personnes, à l'exclusion de toute pratique sportive collective et de toute proximité avec
	d'autres personnes, soit à la promenade avec les seules personnes regroupées dans un
	même domicile, soit aux besoins des animaux de compagnie.
\end{minipage}

\vspace{5mm}

\begin{minipage}{0.1\textwidth}
	\CheckBox[bordercolor=black]{}
\end{minipage}
\begin{minipage}{0.89\textwidth}
	Convocation judiciaire ou administrative.
\end{minipage}

\vspace{5mm}

\begin{minipage}{0.1\textwidth}
	\CheckBox[bordercolor=black]{}
\end{minipage}
\begin{minipage}{0.89\textwidth}
	Participation à des missions d'intérêt général sur demande de l'autorité administrative.
\end{minipage} \\

\TextField[width=8cm,borderwidth=0pt]{Fait à :} \\
\TextField[width=4cm,borderwidth=0pt]{Le :} 
\TextField[width=1cm,borderwidth=0pt,maxlen=2]{à} 
\TextField[width=1cm,borderwidth=0pt,maxlen=2]{h} \\
(Date et heure de début de sortie à mentionner obligatoirement) \\

\TextField[width=14cm,borderwidth=0pt]{Signature :} 

\end{Form}

\end{document}