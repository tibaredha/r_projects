
\section{abstaract}
Dans le cadre du système d’information actif mis
en place par le service d’épidémiologie sur la mortalité
hospitalière au CHU de Blida (Algérie), une étude a été
réalisée pour apprécier l’importance et l’évolution de la
mortalité néonatale enregistrée au CHU au cours des années
1999-2006, ainsi que celles des causes du décès néonatal.\\
La Classification internationale des maladies (CIM-9) a
été utilisée pour coder la nature de la maladie causale. Les
opérations de saisie, de contrôle et d’analyse ont été effectuées
par l’utilisation du logiciel ÉpiInfo™dans sa sixième version.
Au total, 2 167 décès néonatals ont été enregistrés au CHU
pendant la période d’étude, soit une mortalité proportionnelle
de 25,4 \%. La mortalité néonatale précoce (0-6 jours) a
représenté 83,4 \% de l’ensemble de la mortalité néonatale.
Près des deux tiers des décès néonatals précoces sont
intervenus dans les trois premiers jours de vie. L’évolution
mensuelle du nombre de décès néonatals précoces a dessiné
une tendance significativement à la hausse au cours de la
période d’étude (p < 0,05) sans mise en évidence d’effet
saisonnier. Le rapport de masculinité était pratiquement le
même pour la mortalité néonatale précoce et tardive, respectivement
1,4 et 1,5. La prématurité a représenté 42,1 \% des
causes de décès de lamortalité néonatale précoce, suivie par le
syndrome de détresse respiratoire et les infections ; respectivement
17 et 14,4 \%. Les infections ont représenté, avec une
fréquence relative de 36,2 \%, la cause la plus fréquente pour
la mortalité néonatale tardive. Le taux de mortalité néonatale
précoce au cours de la période d’étude, lorsque celui-ci admet
pour dénominateur le nombre de nouveau-nés admis en
néonatalogie pour exprimer la mortalité de service, était de
15,6 \%. Pendant toute la période d’étude, le taux de mortalité
néonatale précoce, en déduisant les décès survenus parmi
les nouveau-nés transférés, pouvait être estimé à 19,2 pour
1 000 naissances vivantes, tandis que le taux de mortalité
néonatale globale pouvait être estimé à 22,3 pour 1 000 naissances
vivantes. Aucune tendance temporelle significative n’a
été mise en exergue. Le CHU de Blida ne se caractérise pas par
un risque inférieur de mortalité néonatale par rapport à celui
enregistré à l’échelle nationale. Les données du CHU de Blida
contribuent à mesurer le degré d’atteinte d’objectifs fixés par
le Programme national sur la périnatalité.