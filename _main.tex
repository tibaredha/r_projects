% Options for packages loaded elsewhere
\PassOptionsToPackage{unicode}{hyperref}
\PassOptionsToPackage{hyphens}{url}
%
\documentclass[
]{book}
\title{Évolution de la mortalité néonatale}
\author{Redha TIBA}
\date{27 05, 2023}

\usepackage{amsmath,amssymb}
\usepackage{lmodern}
\usepackage{iftex}
\ifPDFTeX
  \usepackage[T1]{fontenc}
  \usepackage[utf8]{inputenc}
  \usepackage{textcomp} % provide euro and other symbols
\else % if luatex or xetex
  \usepackage{unicode-math}
  \defaultfontfeatures{Scale=MatchLowercase}
  \defaultfontfeatures[\rmfamily]{Ligatures=TeX,Scale=1}
\fi
% Use upquote if available, for straight quotes in verbatim environments
\IfFileExists{upquote.sty}{\usepackage{upquote}}{}
\IfFileExists{microtype.sty}{% use microtype if available
  \usepackage[]{microtype}
  \UseMicrotypeSet[protrusion]{basicmath} % disable protrusion for tt fonts
}{}
\makeatletter
\@ifundefined{KOMAClassName}{% if non-KOMA class
  \IfFileExists{parskip.sty}{%
    \usepackage{parskip}
  }{% else
    \setlength{\parindent}{0pt}
    \setlength{\parskip}{6pt plus 2pt minus 1pt}}
}{% if KOMA class
  \KOMAoptions{parskip=half}}
\makeatother
\usepackage{xcolor}
\IfFileExists{xurl.sty}{\usepackage{xurl}}{} % add URL line breaks if available
\IfFileExists{bookmark.sty}{\usepackage{bookmark}}{\usepackage{hyperref}}
\hypersetup{
  pdftitle={Évolution de la mortalité néonatale},
  pdfauthor={Redha TIBA},
  hidelinks,
  pdfcreator={LaTeX via pandoc}}
\urlstyle{same} % disable monospaced font for URLs
\usepackage{color}
\usepackage{fancyvrb}
\newcommand{\VerbBar}{|}
\newcommand{\VERB}{\Verb[commandchars=\\\{\}]}
\DefineVerbatimEnvironment{Highlighting}{Verbatim}{commandchars=\\\{\}}
% Add ',fontsize=\small' for more characters per line
\usepackage{framed}
\definecolor{shadecolor}{RGB}{248,248,248}
\newenvironment{Shaded}{\begin{snugshade}}{\end{snugshade}}
\newcommand{\AlertTok}[1]{\textcolor[rgb]{0.94,0.16,0.16}{#1}}
\newcommand{\AnnotationTok}[1]{\textcolor[rgb]{0.56,0.35,0.01}{\textbf{\textit{#1}}}}
\newcommand{\AttributeTok}[1]{\textcolor[rgb]{0.77,0.63,0.00}{#1}}
\newcommand{\BaseNTok}[1]{\textcolor[rgb]{0.00,0.00,0.81}{#1}}
\newcommand{\BuiltInTok}[1]{#1}
\newcommand{\CharTok}[1]{\textcolor[rgb]{0.31,0.60,0.02}{#1}}
\newcommand{\CommentTok}[1]{\textcolor[rgb]{0.56,0.35,0.01}{\textit{#1}}}
\newcommand{\CommentVarTok}[1]{\textcolor[rgb]{0.56,0.35,0.01}{\textbf{\textit{#1}}}}
\newcommand{\ConstantTok}[1]{\textcolor[rgb]{0.00,0.00,0.00}{#1}}
\newcommand{\ControlFlowTok}[1]{\textcolor[rgb]{0.13,0.29,0.53}{\textbf{#1}}}
\newcommand{\DataTypeTok}[1]{\textcolor[rgb]{0.13,0.29,0.53}{#1}}
\newcommand{\DecValTok}[1]{\textcolor[rgb]{0.00,0.00,0.81}{#1}}
\newcommand{\DocumentationTok}[1]{\textcolor[rgb]{0.56,0.35,0.01}{\textbf{\textit{#1}}}}
\newcommand{\ErrorTok}[1]{\textcolor[rgb]{0.64,0.00,0.00}{\textbf{#1}}}
\newcommand{\ExtensionTok}[1]{#1}
\newcommand{\FloatTok}[1]{\textcolor[rgb]{0.00,0.00,0.81}{#1}}
\newcommand{\FunctionTok}[1]{\textcolor[rgb]{0.00,0.00,0.00}{#1}}
\newcommand{\ImportTok}[1]{#1}
\newcommand{\InformationTok}[1]{\textcolor[rgb]{0.56,0.35,0.01}{\textbf{\textit{#1}}}}
\newcommand{\KeywordTok}[1]{\textcolor[rgb]{0.13,0.29,0.53}{\textbf{#1}}}
\newcommand{\NormalTok}[1]{#1}
\newcommand{\OperatorTok}[1]{\textcolor[rgb]{0.81,0.36,0.00}{\textbf{#1}}}
\newcommand{\OtherTok}[1]{\textcolor[rgb]{0.56,0.35,0.01}{#1}}
\newcommand{\PreprocessorTok}[1]{\textcolor[rgb]{0.56,0.35,0.01}{\textit{#1}}}
\newcommand{\RegionMarkerTok}[1]{#1}
\newcommand{\SpecialCharTok}[1]{\textcolor[rgb]{0.00,0.00,0.00}{#1}}
\newcommand{\SpecialStringTok}[1]{\textcolor[rgb]{0.31,0.60,0.02}{#1}}
\newcommand{\StringTok}[1]{\textcolor[rgb]{0.31,0.60,0.02}{#1}}
\newcommand{\VariableTok}[1]{\textcolor[rgb]{0.00,0.00,0.00}{#1}}
\newcommand{\VerbatimStringTok}[1]{\textcolor[rgb]{0.31,0.60,0.02}{#1}}
\newcommand{\WarningTok}[1]{\textcolor[rgb]{0.56,0.35,0.01}{\textbf{\textit{#1}}}}
\usepackage{longtable,booktabs,array}
\usepackage{calc} % for calculating minipage widths
% Correct order of tables after \paragraph or \subparagraph
\usepackage{etoolbox}
\makeatletter
\patchcmd\longtable{\par}{\if@noskipsec\mbox{}\fi\par}{}{}
\makeatother
% Allow footnotes in longtable head/foot
\IfFileExists{footnotehyper.sty}{\usepackage{footnotehyper}}{\usepackage{footnote}}
\makesavenoteenv{longtable}
\usepackage{graphicx}
\makeatletter
\def\maxwidth{\ifdim\Gin@nat@width>\linewidth\linewidth\else\Gin@nat@width\fi}
\def\maxheight{\ifdim\Gin@nat@height>\textheight\textheight\else\Gin@nat@height\fi}
\makeatother
% Scale images if necessary, so that they will not overflow the page
% margins by default, and it is still possible to overwrite the defaults
% using explicit options in \includegraphics[width, height, ...]{}
\setkeys{Gin}{width=\maxwidth,height=\maxheight,keepaspectratio}
% Set default figure placement to htbp
\makeatletter
\def\fps@figure{htbp}
\makeatother
\setlength{\emergencystretch}{3em} % prevent overfull lines
\providecommand{\tightlist}{%
  \setlength{\itemsep}{0pt}\setlength{\parskip}{0pt}}
\setcounter{secnumdepth}{5}
\usepackage{booktabs}
\ifLuaTeX
  \usepackage{selnolig}  % disable illegal ligatures
\fi
\usepackage[]{natbib}
\bibliographystyle{plainnat}

\begin{document}
\maketitle

{
\setcounter{tocdepth}{1}
\tableofcontents
}
\hypertarget{about}{%
\chapter{About}\label{about}}

\begin{Shaded}
\begin{Highlighting}[]
\DecValTok{1}\SpecialCharTok{+}\DecValTok{1}
\end{Highlighting}
\end{Shaded}

\begin{verbatim}
## [1] 2
\end{verbatim}

\hypertarget{ruxe9sumuxe9}{%
\chapter{Résumé}\label{ruxe9sumuxe9}}

\citet{tibaredha}

Dans le cadre du système d'information actif mis
en place par le service d'épidémiologie sur la mortalité
hospitalière au CHU de Blida (Algérie), une étude a été
réalisée pour apprécier l'importance et l'évolution de la
mortalité néonatale enregistrée au CHU au cours des années
1999-2006, ainsi que celles des causes du décès néonatal.
La Classification internationale des maladies (CIM-9) a
été utilisée pour coder la nature de la maladie causale. Les
opérations de saisie, de contrôle et d'analyse ont été effectuées
par l'utilisation du logiciel ÉpiInfo™dans sa sixième version.
Au total, 2 167 décès néonatals ont été enregistrés au CHU
pendant la période d'étude, soit une mortalité proportionnelle
de 25,4 \%. La mortalité néonatale précoce (0-6 jours) a
représenté 83,4 \% de l'ensemble de la mortalité néonatale.
Près des deux tiers des décès néonatals précoces sont
intervenus dans les trois premiers jours de vie. L'évolution
mensuelle du nombre de décès néonatals précoces a dessiné
une tendance significativement à la hausse au cours de la
période d'étude (p \textless{} 0,05) sans mise en évidence d'effet
saisonnier. Le rapport de masculinité était pratiquement le
même pour la mortalité néonatale précoce et tardive, respectivement
1,4 et 1,5. La prématurité a représenté 42,1 \% des
causes de décès de lamortalité néonatale précoce, suivie par le
syndrome de détresse respiratoire et les infections ; respectivement
17 et 14,4 \%. Les infections ont représenté, avec une
fréquence relative de 36,2 \%, la cause la plus fréquente pour
la mortalité néonatale tardive. Le taux de mortalité néonatale
précoce au cours de la période d'étude, lorsque celui-ci admet
pour dénominateur le nombre de nouveau-nés admis en
néonatalogie pour exprimer la mortalité de service, était de
15,6 \%. Pendant toute la période d'étude, le taux de mortalité
néonatale précoce, en déduisant les décès survenus parmi
les nouveau-nés transférés, pouvait être estimé à 19,2 pour

1 000 naissances vivantes, tandis que le taux de mortalité
néonatale globale pouvait être estimé à 22,3 pour 1 000 naissances
vivantes. Aucune tendance temporelle significative n'a
été mise en exergue. Le CHU de Blida ne se caractérise pas par
un risque inférieur de mortalité néonatale par rapport à celui
enregistré à l'échelle nationale. Les données du CHU de Blida
contribuent à mesurer le degré d'atteinte d'objectifs fixés par
le Programme national sur la périnatalité.

\hypertarget{mots-cluxe9s}{%
\section{Mots clés}\label{mots-cluxe9s}}

Mortalité néonatale précoce ·
Mortalité néonatale tardive · Mortalité proportionnelle ·
Causes du décès néonatal · Hôpital · Blida · Algérie ·
Maghreb · Afrique du Nord

\hypertarget{abstract}{%
\section{Abstract}\label{abstract}}

Within the framework of the active information
system set up by the department of epidemiology on hospital
mortality at the Blida (Algeria) University Teaching Hospital
(CHU), a study was carried out to assess the importance and
evolution of neonatal mortality recorded at the CHU in the
last eight years (1999-2006) as well as the causes of neonatal
death. The International Classification of Diseases (ICD-9)
was used to encode the nature of the causal disease. Using the
software EpiInfo™in its sixth version performed data entry,
monitoring and analysis. On the whole, 2,167 neonatal
deaths were recorded at the CHU during the study period,
representing a proportional mortality of 25.4\%. Early
neonatal mortality (0-6 days) accounted for 83.4\% of all
neonatal mortality. Nearly two thirds of early neonatal deaths
occurred in the first three days of life. The monthly evolution
of the number of early neonatal deaths revealed a significant
rising trend during the study period (P \textless{} 0.05) without
identification of seasonal effect. The sex ratio was practically
the same for early and late neonatal mortality, respectively
1.4 and 1.5. Prematurity accounted for 42.1\% of the deaths
in early neonatal deaths, followed by respiratory distress
syndrome and infection, respectively 17.0 and 14.4\%.
Infections, with a relative frequency of 36.2\%, represented

the most common cause for the late neonatal mortality. The
rate of early neonatal mortality during the study period, when
this one took for denominator the number of newborns
admitted in neonatology to express the mortality of service,
was 15.6\%. Throughout the study period, the rate of early
neonatal mortality, without counting the deaths among
transferred newborns, could be estimated at 19.2 per 1,000
live births, while the overall neonatal mortality rate could
be estimated at 22.3 per 1,000 live births. No significant
temporal tendency was pointed out. The CHU of Blida is not
characterized by a lower risk of neonatal mortality compared
to that recorded at national level. The data of the CHU will
contribute to assessing the achievement of objectives set by
the National programme on the perinatality.

\hypertarget{keywords}{%
\section{Keywords}\label{keywords}}

Early neonatal mortality · Late neonatal mortality ·
Proportional mortality · Causes of neonatal death ·
Hospital · Blida · Algeria · Maghreb · Northern Africa

\begin{Shaded}
\begin{Highlighting}[]
\CommentTok{\# par(mar = c(4, 4, .1, .1))}
\CommentTok{\# plot(pressure, type = \textquotesingle{}b\textquotesingle{}, pch = 19)}
\end{Highlighting}
\end{Shaded}

Don't miss Table \ref{tab:nice-tab}.

\begin{Shaded}
\begin{Highlighting}[]
\CommentTok{\# knitr::kable(}
\CommentTok{\#   head(pressure, 10), caption = \textquotesingle{}Here is a nice table!\textquotesingle{},}
\CommentTok{\#   booktabs = TRUE}
\CommentTok{\# )}
\end{Highlighting}
\end{Shaded}

\hypertarget{introduction}{%
\chapter{Introduction}\label{introduction}}

Le taux de mortalité néonatale, calculé comme le nombre
d'enfants décédés entre 0 et 28 jours de vie, rapporté à 1 000
naissances vivantes, pouvait être estimé en Algérie, au milieu
des années 2000, à 25 pour 1 000, représentant 80 \% de la
mortalité infantile. La mortalité néonatale précoce survenant
dans les six premiers jours de la vie était, quant à elle, estimée
à 20 pour 1 000 naissances vivantes, représentant 80 \% de la
mortalité néonatale {[}18{]}.
Le taux de mortalité périnatale (mortinatalité et mortalité
néonatale précoce) est un indicateur remarquable de la
qualité des soins obstétricaux et néonatals. En l'absence de
statistiques systématiques sur les morts foetales pour
déterminer la mortinatalité, la mortalité néonatale précoce
peut être considérée comme un indicateur de la façon dont
sont prodigués les soins aux nouveau-nés dans un établissement.
En effet, la mortalité néonatale figure parmi les
indicateurs de développement d'une collectivité donnée
et constitue le reflet de la qualité des soins obstétricaux et
néonatals dans un établissement de santé.
La mortalité néonatale précoce dans les services de
néonatalogie des hôpitaux des pays pauvres peut frôler
l'hécatombe en dépassant 50 \% {[}4{]}, lorsque le nombre de
décès recensés entre zéro et six jours de vie est ramené au
nombre d'enfants nés vivants admis au sein de ces services.
Dans le cadre du système d'information sur la mortalité
hospitalière mis en place par le service d'épidémiologie
(SEMEP) au CHU de Blida, il était particulièrement
intéressant d'apprécier l'importance et l'évolution de la
mortalité néonatale enregistrée au CHU ainsi que celles des
causes du décès néonatal.

\hypertarget{matuxe9riel-et-muxe9thodes}{%
\chapter{Matériel et méthodes}\label{matuxe9riel-et-muxe9thodes}}

Ce rapport couvre une période de huit ans, de 1999 à 2006,
depuis que le système sur la mortalité hospitalière mis en
place par le SEMEP existe. Tous les décès survenus au CHU
étaient activement recensés par les techniciens du SEMEP au
niveau des différents services, avec l'aide des bureaux des
entrées des trois unités du CHU.
La Classification internationale des maladies (CIM-9) et
ses règles de classement ont été utilisées pour coder la nature
de la maladie causale, tandis que les opérations de saisie, de
contrôle et d'analyse ont été effectuées par l'utilisation du
logiciel Épi-Info™ dans sa sixième version.
Le système sur la mortalité hospitalière mis en place par le
SEMEP permet de déterminer la mortalité proportionnelle
(MP) occasionnée par la mortalité néonatale (nombre de
décès néonatals sur l'ensemble des décès). Le dénominateur
utilisé pour estimer le taux de mortalité néonatale au CHU
provient des données appartenant au registre d'admission du
service de pédiatrie, au registre des naissances vivantes du
service de gynécologie-obstétrique. Celui-ci, avec le service
de pédiatrie, appartient à la même unité de lieu constituée par
le complexe mère-enfant du CHU. Il a par ailleurs été tenu
compte des nouveau-nés transférés des maternités périphériques
ou d'autres hôpitaux.
L'analyse des séries chronologiques a essentiellement fait
appel au coefficient de corrélation (r) ainsi qu'au coefficient
de corrélation des rangs de Spearman (r0). Le coefficient de
corrélation r était déterminé pour apprécier l'évolution
temporelle du nombre mensuel de décès néonatals, tandis
que le coefficient de corrélation r0 était déterminé pour
apprécier la tendance dessinée par le taux annuel de la
mortalité néonatale et de la proportion annuelle des nouveaunés
transférés des maternités périphériques au cours de la
période d'étude. L'analyse des séries chronologiques a
également fait appel à l'analyse de plans factoriels à une
répétition pour déceler un éventuel effet des variables du plan
{[}6{]}. Sauf indication contraire, les moyennes étaient accompagnées
des écarts-types des valeurs des différentes séries.
La prématurité a été définie comme l'état d'un nouveau-né
déclaré né avant la 37e semaine d'aménorrhée d'une
gestation, tandis qu'un cas de décès néonatal précoce a
concerné un nouveau-né déclaré né vivant après une
grossesse de 25 semaines au minimum.

\hypertarget{ruxe9sultats}{%
\chapter{Résultats}\label{ruxe9sultats}}

Au total, 8 541 décès ont été enregistrés au CHU de Blida
de 1999 à 2006, dont 2 167 décès néonatals, soit une MP de
25,4 \%. Autrement dit, un décès sur quatre survenant au
CHU de Blida concernait un nouveau-né âgé de moins
de 29 jours. La MP occasionnée par la mortalité néonatale

occupait la première position et devançait celle des maladies
de l'appareil circulatoire (21 \%) et celle des traumatismes et
empoisonnements (19,8 \%) qui occupaient respectivement
les deuxième et troisième position.
La mortalité néonatale précoce (0-6 jours) avec 1 808
décès a représenté 83,4 \% de l'ensemble de la mortalité
néonatale, tandis que la mortalité néonatale tardive
(7-28 jours), avec 359 décès, a représenté 16,6 \% de
l'ensemble de la mortalité néonatale. Il existait une relation
entre l'année (comme variable servant de base aux
comparaisons) et le type de mortalité : la fréquence relative
de la mortalité néonatale précoce en 1999 (76,8 \%) était plus
basse (p \textless{} 0,02). Mais dans l'ensemble, d'une année à une
autre, la mortalité néonatale précoce représentait plus des
quatre cinquièmes de la mortalité néonatale globale.
La mortalité néonatale proportionnelle dans ses deux
composantes est pratiquement restée stationnaire pendant la
période d'étude, malgré une hausse en 2003 qui a affecté la
mortalité néonatale précoce (Tableau 1).
Parmi les décès néonatals précoces, 24,0 \% sont intervenus
avant le premier jour de vie, tandis que près des deux
tiers (63,0 \%) des décès sont intervenus dans les trois
premiers jours de vie. Pour les décès néonatals tardifs,
37,6 \% sont survenus entre sept et dix jours de vie.
L'évolution annuelle de la répartition des décès néonatals
précoces et tardifs selon l'âge en jours révolus du décès
ne mettait pas en évidence de différence significative
(Tableau 2).

L'évolution mensuelle (8 × 12 = 96 mois) du nombre de
décès néonatals précoces a dessiné une tendance significativement
à la hausse au cours de la période d'étude (r = 0,21,
p \textless{} 0,05), tandis que la mortalité néonatale tardive a dessiné
une tendance à la baisse, mais de façon non significative
(r = --0,18 NS). La résultante en est une tendance à la hausse
de la mortalité néonatale globale au cours de la période
d'étude, mais de façon non significative (r = 0,13 NS).
Pour la mortalité néonatale précoce, l'analyse du plan
factoriel relatif à l'année et au mois de l'année n'a pas mis en
évidence un effet saisonnier (F = 1,34 NS), tandis qu'un
excès de décès était notamment enregistré pour l'année 2003
(F = 4,25 NS, p \textless{} 0,001). L'analyse du plan factoriel relatif
au jour de la semaine et à l'heure du jour n'a mis en évidence
ni un effet d'heure (F = 1,54 NS) ni un effet de jour de
semaine (F = 0,69 NS).
D'une année à une autre, la prématurité a représenté, en
règle générale, plus de 40 \% des causes de décès de la
mortalité néonatale précoce, suivie par le syndrome de
détresse respiratoire et les infections néonatales, respectivement
17 et 14,4 \%.
Les infections ont représenté, pour la mortalité néonatale
tardive, la cause la plus fréquente avec 43,8 \% des causes de
décès en 1999 : cette proportion a ensuite régulièrement
baissé au cours du temps pour représenter, en 2006,
20,6 \% des causes de décès (Tableau 3).
Le rapport de masculinité était de 1,5 pour la mortalité
néonatale globale. Le rapport de masculinité restait

pratiquement le même, aussi bien pour la phase précoce que
pour la phase tardive, respectivement de 1,4 et 1,5.
Cependant, il n'existait pas de relation entre le type
de mortalité et le sexe : la mortalité néonatale précoce
a représenté respectivement 83 et 84 \% de la mortalité
néonatale chez les nouveau-nés de sexe masculin et
ceux de sexe féminin. Les causes de décès chez les deux
sexes ne semblaient pas non plus différer sensiblement
(Tableau 4).

La durée de séjour moyenne des enfants décédés en
période néonatale précoce pendant la période d'étude était
de 1,4 ± 0,0 jour (moyenne ± erreur type), tandis que la
médiane était de 1 jour. D'une année à une autre, cette
moyenne n'a pas varié de façon significative. La durée de
séjour moyenne des enfants décédés en période néonatale
tardive était de 8,5 ± 0,3 jours (médiane = 8 jours) ; la
durée de séjour moyenne en 1999 était relativement basse
(Tableau 5).

Les taux estimés de mortalité néonatale précoce et tardive,
lorsque ces taux admettent comme dénominateur le nombre
de nouveau-nés admis en néonatalogie pour exprimer la
mortalité de service, ont dessiné une tendance significativement
à la baisse expliquant par là même la tendance à la
baisse du taux de mortalité globale. Par contre, aucune
tendance temporelle significative n'était mise en évidence
lorsque les taux sont exprimés en fonction du nombre de
naissances vivantes (Tableau 6).
Le taux de mortalité néonatale précoce pendant la période
d'étude (1999-2006), en prenant en dénominateur le nombre
de nouveau-nés admis en néonatalogie, était de 15,6 \%,
tandis que le taux de mortalité tardive correspondant était de
3,1 \%, soit un taux de mortalité néonatale globale de 18,7 \%.
On pouvait donc raisonnablement estimer que chaque
nourrisson admis en néonatalogie avait presque une
probabilité de 1/6 de décéder avant le septième jour de vie.

Le taux de mortalité néonatale précoce, en prenant comme
dénominateur le nombre de naissances vivantes enregistrées
dans l'établissement, était, pendant la période d'étude, de
22,9 pour 1 000, tandis que le taux de mortalité néonatale
tardive était de 4,5 pour 1 000, soit un taux de mortalité
néonatale globale de 27,4 pour 1 000.
Les nouveau-nés transférés des maternités périphériques
ont représenté 20,4 \% des nouveau-nés admis en néonatalogie
(2 366-11 612) et 2,9 \% des naissances vivantes
(2 366-81 345) au CHU de Blida. Le nombre des décédés
parmi les transférés s'est élevé à 410, soit une proportion de
17,3 \% (410-2 366).
Le taux de mortalité néonatale globale au CHU de Blida,
pendant toute la période d'étude, pouvait alors, en soustrayant
les décès parmi les transférés, être plus justement estimé à
22,3 pour 1 000 naissances vivantes (1 757-78 879), soit une
diminution de 18,9 \% par rapport au taux qui ne tiendrait

pas compte des décès survenus parmi les nouveau-nés
transférés.
Pendant toute la période d'étude, le taux de mortalité
néonatale précoce, en déduisant les décès survenus parmi les
nouveau-nés transférés (294 décès), pouvait être estimé à
19,2 pour 1 000 naissances vivantes, soit une diminution
de 16,2 \% par rapport au taux qui ne tiendrait pas compte de
ces décès.
De même, le taux de mortalité néonatale tardive, en
retranchant les décès parmi les transférés (116 décès),
pouvait être estimé pendant toute la période d'étude à 3,1
pour 1 000 naissances vivantes, soit une diminution de
31,1 \% par rapport au taux qui ne tiendrait pas compte de ces
décès.

\hypertarget{discussion}{%
\chapter{Discussion}\label{discussion}}

La mortalité néonatale a donc constitué un quart des décès
survenant au CHU de Blida pendant la période d'étude. Une
MP de cet ordre (28 \%) a été retrouvée au CHU d'Annaba
(Algérie) au début de la décennie écoulée {[}8{]}. Il est frustrant
de ne pouvoir pousser plus loin les comparaisons, car les
tentatives visant à établir des statistiques de mortalité et de
morbidité avec des méthodes standardisées sont restées très
limitées, même à l'échelle des CHU {[}7{]}.
Les décès néonatals survenus en dehors du service de
néonatalogie ont constitué moins de 4 \% (3,3 \%) de
l'ensemble des décès néonatals survenus au CHU. Du fait
de cette faible proportion, ces décès ont été comptabilisés
avec ceux qui sont survenus en néonatalogie pour calculer
la mortalité de service et pour déterminer les autres taux de
mortalité néonatale.
La mortalité néonatale précoce a représenté plus de 80 \%
de la mortalité néonatale au CHU de Blida : elle semblait du
même ordre de grandeur que celle enregistrée dans la plupart
des pays de la sous-région ouest africaine où elle pouvait
constituer 75 à 90 \% de la mortalité néonatale {[}2,10{]}. Le
risque mesurant la mortalité de service était aussi identique à
celui enregistré dans un hôpital de Côte-d'Ivoire {[}19{]}.
Les comparaisons, malgré les difficultés reconnues
qu'elles suscitent du point de vue de la validité des données
et de la standardisation des définitions {[}11{]}, sont encore plus
intéressantes lorsque le taux de mortalité néonatale admet
comme dénominateur le nombre de naissances vivantes
enregistré dans l'établissement.
Le taux moyen de mortalité néonatale précoce à l'hôpital
de gynécologie-obstétrique de Hanoi pendant la période
1991-1995, de l'ordre de 24 pour 1 000 naissances vivantes
{[}13{]}, était approximativement identique à celui enregistré au
CHU de Blida. Au CHU de Dakar, pendant la décennie
1994-2003, ce taux était encore plus élevé : 45,5 pour
1 000 et augmentait même à 67,5 pour 1 000 lorsqu'il était

tenu compte des nouveau-nés transférés des maternités
périphériques et dont la proportion représentait 4,2 \% {[}9{]}.
Le taux de mortalité néonatale précoce à l'hôpital de
gynécologie-obstétrique de Hanoi n'était plus, en 1995, que
de 15,3 pour 1 000, chute essentiellement due à une
augmentation de la proportion d'accouchements normaux
et à une diminution drastique de la proportion des cas
adressés par les maternités périphériques, suite à un
changement dans la politique nationale de santé intervenu
en 1992 {[}13{]}. Ce taux restait néanmoins élevé lorsqu'il était
comparé à celui enregistré en 1991 dans un hôpital
universitaire de Malaisie : 5,5 pour 1 000 naissances
vivantes {[}3{]}. Une baisse du taux de mortalité néonatale a
aussi été enregistrée dans un hôpital universitaire nigérian, de
50 pour 1 000 en 1974 à 16 pour 1 000 en 1981, baisse
expliquée par une réduction de la mortalité chez les enfants
dont le poids de naissance dépassait 2 500 g et ceux
souffrant d'asphyxie {[}21{]}. Le taux de mortalité néonatale
précoce a atteint 34 pour 1 000 dans un centre universitaire
de Madagascar {[}5{]}.
Le taux de mortalité néonatale précoce pouvait être
cependant bas en Afrique, de l'ordre de 14 pour 1 000 dans
un district sanitaire au Burkina Faso, tandis que le taux de
mortalité néonatale globale était de 24 pour 1 000 {[}25{]}.
Le taux de mortalité néonatale précoce dans un district
rural au Mali était aussi faible (17 pour 1 000), tandis que le
taux de mortalité néonatale tardive était de 7 pour 1 000, soit
un taux de mortalité néonatale globale de 24 pour 1 000, plus
faible que le taux national estimé à 57 pour 1 000 {[}24{]}.
Des taux nationaux du même ordre de grandeur pouvaient
être retrouvés pour l'Afrique et l'Asie. Pour l'Afrique au sud
du Sahara, le taux moyen régional en 1995 était estimé à
39 pour 1 000 (variant de 13 pour 1 000 au Kenya à 108 pour
1 000 au Sénégal), tandis que pour l'Asie du Sud, pour la
même année, le taux de mortalité néonatale variait de 42 à
57 pour 1 000 {[}14{]}.
Le taux de mortalité néonatale précoce, à partir des
données portant sur 7 993 grossesses dans six pays en
développement, pouvait encore être plus faible : 9 pour
1 000 naissances vivantes {[}20{]}. Au Québec, pendant la
période 1976-2005, le taux de mortalité néonatale précoce
n'a été que de 2,9 pour 1 000, alors que le taux de mortalité
néonatale global était de 3,4 pour 1 000 {[}15{]}. En République
tchèque, au milieu des années 1990, le taux de mortalité
néonatale global n'était plus que de 3,8 pour 1 000 {[}16{]}.
On peut à ce niveau mettre en exergue que le CHU de
Blida ne se distingue finalement pas par un risque moins
élevé de mortalité néonatale, puisqu'il enregistre des taux
similaires à ceux enregistrés au niveau national {[}18{]}. Par
ailleurs, le risque annuel de décéder pour un nouveau-né,
exprimé pour 1 000 naissances vivantes, est resté stationnaire
pendant toute la période d'étude, malgré la tendance à la
baisse enregistrée pour la mortalité correspondante de

service. Il est vrai que la proportion des nouveau-nés
transférés des maternités périphériques était relativement
élevée ; un nouveau-né sur cinq. Cependant, la proportion
annuelle de ces transférés a même dessiné une tendance à la
baisse (r0 = --0,88, p = 0,01), passant de 30,1 \% en 2000
(proportion la plus élevée enregistrée) à 17,5 \% en 2006
(proportion la plus basse enregistrée), suggérant que la
fonction de centre de référence des grossesses à risque pour
la région dévolue au CHU n'a pas été accrue au cours de
la période d'étude. De fait, cette période n'a été caractérisée
ni par l'acquisition de nouveaux équipements, ni par la mise
en place de nouvelles techniques d'exploration ou de soins,
à l'exception d'efforts consentis en matière d'hygiène
hospitalière.
La surmortalité masculine en matière de décès néonatals
est constatée dans presque toutes les recherches antérieures,
avec un sex-ratio par exemple de 1,3 pour les décès précoces
dans un centre hospitalier universitaire malgache {[}5{]}, et a été
aussi retrouvée au Maroc {[}1{]}.
Parmi les causes de mortalité néonatale précoce identifiées
à travers le monde, la prématurité est souvent retrouvée en
première position. Il est possible, dans notre série, que
quelques faux cas de prématurité aient été inclus, notamment
des hypotrophiques à terme d'un poids de naissance inférieur
à 2 500 g (date des dernières règles inconnue ou douteuse,
absence de contrôle par une échographie précoce et nondétermination
à la naissance de l'âge gestationnel par la
morphologie et l'examen neurologique). Malgré ces réserves,
on peut considérer que la prématurité a représenté
pendant la période d'étude plus de 40 \% des décès néonatals
précoces. Une proportion supérieure à 70 \% a été retrouvée
à l'hôpital de gynécologie-obstétrique de Hanoi {[}13{]} et au
CHU de Lomé, au Togo, au début des années 1990 {[}4{]}. Une
proportion de l'ordre de 50 \% a été retrouvée à l'institut de
nutrition et de santé en Guinée {[}10{]}.
Il semble bien que la prématurité puisse représenter plus
de 60 \% des causes de décès néonatals précoces {[}20{]}.
Cependant, cette contribution peut être plus faible : de
l'ordre de 23 \% dans un district sanitaire au Burkina
Faso {[}25{]}.
La prématurité a souvent été décrite comme une « maladie
à caractère social très prononcé », car la prématurité est plus
fréquente parmi les enfants de femmes pauvres et peu
éduquées {[}13{]}. Le taux de prématurité (naissance avant
37 semaines de gestation), qui varierait de 5 à 9 \% dans les
pays en développement {[}23{]}, reste en réalité du même ordre
de grandeur dans les pays développés et dépasserait 12 \%
aux États-Unis {[}12{]}.
La fréquence de la prématurité reste aussi élevée et ne
semble pas baisser, malgré les avancées réalisées dans la
connaissance des facteurs de risque et l'introduction de
mesures médicales pour lutter contre ce problème de santé
publique. Cependant, des progrès considérables ont été

enregistrés en matière de survie des nouveau-nés prématurés
{[}12,17{]}.
La proportion des prématurés décédés parmi les nouveaunés
pourrait être estimée à environ 1 \% dans notre série
(844-78 979), mais la létalité due à la prématurité ne peut être
estimée, étant donné que le nombre total des prématurés nés
au CHU n'a pu être établi. La mortalité élevée chez les
prématurés est due au fait que les nouveau-nés prématurés
sont très vulnérables aux risques d'asphyxie et d'infection
par immaturité des fonctions immunitaire, respiratoire, etc.
L'infection a aussi été identifiée comme une cause
majeure de mortalité néonatale, dépassant la contribution
de la prématurité, et pouvant représenter jusqu'à 50 \% des
causes de la mortalité néonatale {[}25{]}. Dans ce district
sanitaire, l'infection a représenté la première cause de
mortalité pendant les deux phases de la période néonatale,
mais était plus impliquée, comme au CHU de Blida, dans des
proportions différentes il est vrai, dans la phase tardive que
dans la phase précoce. Dans notre série, les infections
nosocomiales ont probablement constitué l'essentiel des
infections en surpassant le poids des infections d'origine
maternofoetale, mais le problème n'a pu être quantifié. Les
efforts consentis en matière d'hygiène hospitalière pourraient
expliquer les pourcentages relativement bas enregistrés en
2006, concernant la contribution des infections dans la
survenue de décès néonatals.
Les malformations congénitales ont constitué 11,1 \% de
l'ensemble des décès néonatals. Nonobstant les nouveau-nés
transférés des maternités périphériques et qui seraient
atteints d'anomalies congénitales, les malformations n'ont
constitué que 0,3 \% des naissances vivantes. Le même taux
a été retrouvé à l'hôpital de gynécologie-obstétrique de
Hanoi {[}13{]}, mais ces taux sont probablement sous-estimés,
car le diagnostic est clinique, et les malformations mineures
ne sont pas rapportées. Il est parfois reconnu, parmi les
décès néonatals précoces, qu'un nouveau-né sur deux
décède dans les premières 24 heures de vie {[}18,13{]}. À
Blida, cette proportion était de 25 \% environ, mais dépassait
55 \% lorsqu'elle était mesurée dans les 48 premières
heures de vie.
Tous les décès néonatals ne sont certainement pas
évitables. Mais la moitié d'entre eux pourraient être évités
grâce à des mesures simples et peu coûteuses. Des pays
comme la Colombie ou le Sri Lanka qui enregistrent moins
de 15 décès pour 1 000 naissances vivantes ont fait la preuve
que des techniques coûteuses ne sont pas un préalable à une
réussite en la matière. Dans la plupart des pays, la mortalité
des enfants dont la mère a bénéficié de soins prénatals et de
soins qualifiés à l'accouchement est généralement inférieure
de plus de la moitié à la mortalité des enfants dont la mère n'a
pas bénéficié de ce type de soins, même si le problème de
la fragmentation actuelle des soins au nouveau-né reste
posé {[}22{]}.

\hypertarget{conclusion}{%
\chapter{Conclusion}\label{conclusion}}

L'Algérie a pris conscience du lourd fardeau représenté par la
mortalité néonatale en mettant en place un Programme
triennal national sur la périnatalité. La surveillance de la
grossesse et du travail, d'une part, et la prise en charge
immédiate du nouveau-né par une équipe obstétricopédiatrique,
d'autre part, sont les deux piliers de ce programme,
dont des objectifs quantifiés concernent, par exemple, une
réduction de 25 \% de la mortalité néonatale précoce et une
réduction de 30 \% de la mortalité des nouveau-nés de faible
poids de naissance d'ici 2009 {[}18{]}.
Les données du CHU de Blida sur la mortalité néonatale,
décrivant une période fort homogène, devraient contribuer
à mesurer le degré d'atteinte d'objectifs fixés de ce
programme, d'autant plus que le risque de mortalité
néonatale déterminé pour le CHU rejoint celui enregistré à
l'échelle nationale.

This \texttt{gitbook} uses the same social sharing data across all chapters in your book- all links shared will look the same.

Specify your book's source repository on GitHub using the \texttt{edit} key under the configuration options in the \texttt{\_output.yml} file, which allows users to suggest an edit by linking to a chapter's source file.

Read more about the features of this output format here:

\url{https://pkgs.rstudio.com/bookdown/reference/gitbook.html}

Or use:

\begin{Shaded}
\begin{Highlighting}[]
\NormalTok{?bookdown}\SpecialCharTok{::}\NormalTok{gitbook}
\end{Highlighting}
\end{Shaded}


  \bibliography{book.bib,packages.bib}

\end{document}
