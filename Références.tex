\section{references}
1. Abourmane EH (2006) Étude rétrospective de la mortalité
périnatale au niveau de la maternité de l’hôpital * Essalama *
d’El-Kelâa-des-Sraghna : causes et circonstances, Mémoire présenté
pour l’obtention du diplôme de maîtrise en administration
sanitaire et santé publique. Promotion 2004-2006. Institut national
d’administration sanitaire, Rabat


2. Alihonou E, Dan V, Ayivi B, et al (1991) Mortalité néonatale au
Centre national hospitalier et universitaire de Cotonou : incidence,
causes et moyens de lutte. Med Afr Noire 38:745–1


3. Al Mohdzar SA, Haque E, Abdullah WA (1993) Changes of
perinatal statistics in a semi-urban set-up between two time
periods in Malaysia. Asia Oceania J Obstet Gynaecol 19:401–5


4. Balaka B, Agbere AD, Kpemissi E, et al (1998) Évolution de la
mortalité néonatale précoce en dix ans (1981–1982 et 1991–1992)
au CHU de Lomé : quelle politique de santé néonatale pour
demain ? Med Afr Noire 45:430–4


5. Berthin A (2004) Facteurs de mortalité néonatale précoce au
centre hospitalier universitaire de gynécologie-obstétrique de
Befelatanana, mémoire de fin d’études. Institut national de la santé
publique et communautaire. Madagascar


6. Bezzaoucha A (2003) Séries chronologiques. In: Compléments en
techniques épidémiologiques de base. Office des publications
universitaires, Alger, pp. 183–210


7. Bezzaoucha A, Lamdjadani N, Makhlouf F (1994) Le système
d’information sur la morbidité au CHU de Bab-El-Oued :
description et résultats. J Alger Med 4:259–63


8. Bounecer H, Bachtarzi T, Gherbi M (1992) Mortalité et morbidité
au CHU de Annaba. Santé plus 11:24–5	
	

9. Cissé CT, Yacoubou Y, Ndiaye O, et al (2006) Évolution de la
mortalité néonatale précoce entre 1994 et 2003 au CHU de Dakar.
J Gynecol Obstet Biol Reprod (Paris) 35:46–52


10. Diallo S, Boye Camara Y, Mamady D, et al (2000) Mortalité
infantojuvénile à l’Institut de nutrition et de santé de l’enfant
(Inse). Med Afr Noire 47:516–9


11. Draper ES, Field DJ (2007) Epidemiology of prematurity: how
valid are comparisons of neonatal outcomes? Semin Fetal
Neonatal Med 12:337–43


12. Goldenberg RL, Culhane JF, Iams JD, Romero R (2008)
Epidemiology and causes of preterm birth. Lancet 371:75–84


13. Hoan PT, Bao TV, Phong DN, et al (2000) Mortalité néonatale
précoce à l’hôpital de gynécologie-obstétrique de Hanoï, Vietnam.
Bull Soc Pathol Exot 93:62–5. http://www.pathexo.fr/pages/
articles/2000/2000-T93-1/2082.html


14. Hyder AA, Wali SA, McGuckin J (2003) The burden of disease
from neonatal mortality: a review of South Asia and Sub-Saharan
Africa. BJOG 110:894–901


15. Institut de la statistique du Québec (2006) Taux de mortinatalité,
de mortalité périnatale, néonatale et infantile, Québec :
1976–2005. Gouvernement du Québec


16. Koupilova I, McKee M, Holcik J (1998) Neonatal mortality in the
Czech Republic during the transition. Health Policy (New York)
46:43–52


17. Lams JD, Romero R, Culhane JF, Goldenberg RL (2008) Primary,
secondary, and tertiary interventions to reduce the morbidity and
mortality of preterm birth. Lancet 371:164-75


18. Lebane D, Aït Ouyahia B, Vert P, Breart G (2006) Programme
national périnatalité (Programme triennal 2006-2009), ministère
de la Santé, de la Population et de la Réforme hospitalière, Agence
nationale de documentation sur la santé, Alger


19. Mutombo T (1993) Mortalité néonatale dans un hôpital rural : cas
de l’hôpital protestant de Dabou (Côte-d’Ivoire). Med Afr Noire
40:471–9


20. Ngoc NT, Merialdi M, Abdel-Aleem H, et al (2006) Causes of
stillbirths and early neonatal deaths: data from 7993 pregnancies
in six developing countries. Bull World Health Organ 84:
699–705


21. Okolo AA, Omene JA (1985) Trends in neonatal mortality in
Benin City. Nigeria, Int J Gynaecol Obstet 23:191–5
22. Organisation mondiale de la santé (2005) Donnons sa chance à
chaque mère et à chaque enfant. Rapport sur la santé dans le
monde OMS, Genève


23. Papiernik E (2003) Prévention de la prématurité, In: Pons JC,
Goffinet F (eds) Traité d’obstétrique, Flammarion Médecine–
Sciences, Paris, pp. 389–401


24. Sidibe T, Sangho H, Doumbia S, et al (2006) Mortalité
néonatale dans le district sanitaire de Kolokani. J Pediatr Pueric
19:272–6


25. Sombie I (2001) Mortalité néonatale et maternelle en milieu
rural au Burkina Faso, année 2001 – Indicateurs de base pour
un programme d’intervention pour une maternité à moindre
risque dans le district sanitaire de Houndé, mémoire pour
l’obtention du diplôme d’études approfondies en sciences de
la santé. Université libre de Bruxelles, faculté de médecine
et de pharmacie, école de santé publique : année scolaire
2001–2002	